\chapter{State of Art} \label{chapter:state_of_art}

The increasing complexities in managing \ac{lf} items spotlight the limitations of traditional approaches and the potential for intelligent solutions. This chapter explores the evolution of \ac{lfms}, focusing on manual methods, inventory management principles and, ultimately, the most relevant \ac{lf} platforms and solutions. By examining existing solutions, challenges, and best practices, this chapter lays the groundwork for designing a sturdy and efficient intelligent management system tailored to the demands of modern users.

%% ____________________ Manually Managing LF Items ____________________ %%

\section{Traditionally Managing \acl{lf} Items} \label{sec:manually-managing-lf-items}

% \subsection{Traditional Management Systems} \label{subsec:traditional-lf-systems}

Traditional \acp{lfms} typically relied on manual processes to log and manage items, sometimes papers or books and, in a few cases, spreadsheets \cite{Anas2023}. The fundamental components included physical logs, where details of found items - such as descriptions, location, and date of discovery - are recorded by staff or custodians.

Often, the responsibility of maintaining these records fell to a designated individual or department. Items are categorised and stored in a secure area, with the hope that owners would reclaim them. Matching found items to reported losses was a manual process, requiring significant time and effort \cite{Anas2023}. Descriptions provided by claimants are cross-referenced with recorded details to determine ownership. In some cases, rudimentary tagging systems are used to label items, aiding the identification process.

In environments like universities or corporate campuses, basic digital tools such as spreadsheets are usually introduced to track items. However, the overall workflow remains heavily dependent on manual oversight. Community bulletin boards, notices, or word-of-mouth are also standard methods of informing individuals about found items.

Despite their simplicity, these systems played a critical role in facilitating the return of lost belongings in the pre-digital era \cite{Mayura2024}. They promoted a sense of trust and collaboration within communities, relying on the goodwill and honesty of both finders and administrators. The traditional systems established the groundwork for modern approaches, providing valuable insights into the challenges and requirements of effective \ac{lf} management.

Despite the previously detailed inefficiencies of such systems, there are still those who would find these manual workflows to be more trustworthy. For some, the human element brings a level of accountability and understanding that automated systems cannot replicate. Manual processes allow for subjective judgment, which can be beneficial in complex scenarios where nuance is required. The tactile nature of handling paper documents or physical records stimulates a sense of security and reliability. Moreover, people who have had negative experiences with technology might prefer traditional methods, viewing them as more stable and less prone to glitches or failures.

%% ____________________ Challenges of Traditional LFMS ____________________ %%

While traditional \acp{lfms} have historically served their purpose in smaller or less demanding contexts, they face significant challenges when scaled to handle larger volumes of items or more complex environments \cite{Mayura2024}. The scalability of such systems is inherently constrained by their reliance on manual processes and both limited technological integration and automation.

A key obstacle is the dependency on human effort. As the number of items increases, so does the burden on the responsible staff, leading to delays, errors, and inefficiencies. In high-traffic environments, the sheer volume of items can quickly overwhelm even the most organised traditional systems \cite{Yao2019}. Without automation, processing and resolving claims becomes a time-intensive task, reducing the workflow's overall effectiveness. 

Another barrier to scaling is the lack of centralised data management. In traditional systems, records are often siloed, with each location or department maintaining its logs \cite{Soumya2024}. Especially in larger organisations or distributed campuses, the absence of a unified database also impedes efficient reporting and analysis of trends, such as identifying frequent loss locations or categories of items \cites{Sinha2024, Soumya2024}.

Communication between stakeholders presents further challenges. Traditional systems often lack any mechanisms for notifying individuals about found items or updating claimants on the status of their reports, resulting in unclaimed items \cite{Zhou2023} and frustrations, particularly in large-scale operations where the number of inquiries can be substantial.

Finally, the security of manual systems poses significant concerns. As the volume of items increases, the risk of theft, loss, or unauthorised access also rises. Inadequate labelling and verification processes can lead to disputes or errors in returning items to their rightful owners, further eroding trust in the system \cite{Guinard2008}.

% These obstacles highlight traditional systems' limitations in adapting to the demands of modern \ac{lf} management. While effective in smaller, community-focused settings, their scalability issues underscore the need for innovative solutions that leverage technology to enhance efficiency, security, and user satisfaction.

%% ____________________ Inventory Management Systems ____________________ %%

\section{\aclp{ims}} \label{sec:ims}

\acp{ims} are tools designed to manage and track inventory levels across various domains. These systems enable organisations to streamline their operations, reduce costs, and enhance customer satisfaction by securing the availability of necessary items at the right time and place \cite{Pauliina2024}. An \ac{ims} employs methodologies to identify and classify inventory items based on details like quantity, volume, value or variability, optimising inventory performance through targeted strategies \cite{Pauliina2024}.

Fundamentally, an \ac{ims} aims to address some challenges, such as maintaining inventory levels, reducing holding costs, mitigating risks and many others associated with collections and stocks. Techniques like safety stock determination and lot-sizing methods have been widely adopted \cite{Prabakaran2023}. Such systems have historically evolved to incorporate some technological advancements, obtaining the capability to predict, plan, and execute inventory operations automatically and effectively \cite{Chebet2019}.

While conventional \acp{ims} primarily focus on physical products, their underlying principles can be effectively applied to tracking and managing \ac{lf} items. This intersection becomes particularly relevant in environments where the inventory (in this case, lost items) exhibits variability in value, volume, and retrieval demand. Inventory management principles, such as classification and optimisation, can be applied to \acp{lfms} to streamline processes and enhance user experiences. For instance, using the classification method mentioned before, in a \ac{lfms}, items could be categorised using an adapted ABC-XYZ\footnote{Suryaputri Z. and Gabriel, D.S. and Nurcahyo R., Integration of ABC-XYZ Analysis in Inventory Management Optimization: A Case Study in the Health Industry, Proceedings of the International Conference on Industrial Engineering and Operations Management, 2020, \url{https://ieomsociety.org/proceedings/2022nigeria/70.pdf}}framework that enables prioritisation of storage, retrieval, and notification efforts \cite{Khobragade2018}. The Table \ref{tab:abc_xyz} illustrates a potential result of that association:

\begin{table}[H]
\centering
\caption{Classification of \ac{lf} items using the ABC-XYZ framework}
\begin{tabular}{|c|c|c|c|c|}
\hline
\textbf{Class} & \textbf{Value} & \textbf{Volume} & \textbf{Retrieval Demand} & \textbf{Examples} \\ 
\hline
A-X & High & Low & Consistent & Electronic devices, jewellery \\ 
\hline
B-Y & Moderate & Moderate & Fluctuating & Wallets, bags, clothing \\ 
\hline
C-Z & Low & High & Irregular & Umbrellas, stationery \\ 
\hline
\end{tabular}
\label{tab:abc_xyz}
\end{table}

The insights from \citeauthoryear{Plinere2016} further support the integration of these methodologies into \acp{lfms}. Their case study emphasises the significance of structured inventory management practices for improving operational efficiency. For instance, high-priority items, such as class A-X items, can benefit from focused attention and expedited claim processes, guaranteeing user satisfaction while reducing storage costs. Furthermore, their findings highlight the value of predictive analytics in addressing slow-moving or stagnant inventory, matching to unclaimed items \cite{Plinere2016}. Technological integration, as illustrated in the case study, offers another avenue for improvement. The use of \ac{rfid} and \ac{qr} codes in conventional inventory systems ensures accurate tracking and categorisation of items \cite{Plinere2016, Sohail2018}. Moreover, the adoption of resource planning principles, such as imposing a centralised warehouse, would improve transparency and decision-making, possibly enabling a seamless synchronisation of data across departments. Lastly, the \citeauthoryear{Plinere2016} study underlines the importance of standardisation in inventory management policies, e.g., uniform intake procedures, categorisation standards, and clear guidelines for item disposition.

Adapting \acp{ims} for \ac{lf} management would definitely address some unique challenges \cite{Pauliina2024}. Unlike traditional inventory, lost items often have sentimental value or urgent retrieval needs, requiring the system to incorporate real-time tracking and user-friendly interfaces. Moreover, integrating predictive analytics, commonly used in \acp{ims} for forecasting demand, would be leveraged to anticipate peak periods of item loss or retrieval, e.g., events or seasons that may influence the volume and types of items lost, allowing proactive resource allocation \cite{Prabakaran2023}.

Building on these foundational principles, open-source \acp{ims} have evolved to address specific needs, blending traditional methodologies with modern technologies to enhance their adaptability and utility across domains. Odoo Inventory\footnote{\url{https://www.odoo.com/app/inventory}}, for instance, exemplifies this progression by integrating inventory management with enterprise-level tools like sales and customer relationship management. Its multi-warehouse support and barcode scanning capabilities make it a rich solution for medium to large enterprises. In contrast, Snipe-IT\footnote{\url{https://snipeitapp.com/demo}} focuses on a narrower domain (asset tracking), offering a well-designed interface for managing fixed assets, albeit without features like demand forecasting or multi-warehouse management. Moving further into specialised territories, InvenTree\footnote{\url{https://inventree.org/}} demonstrates how inventory systems can cater to engineering and manufacturing by providing tools for managing parts and components through hierarchical structures and batch tracking. However, it lacks the scalability offered by integration with other systems. Meanwhile, SkuNexus\footnote{\url{https://skunexus.com}} exemplifies the pivot toward e-commerce needs, combining advanced reporting, omnichannel order fulfilment, and process automation for complex operations, though its configurability demands significant effort during implementation. For small-scale, niche applications, PartKeepr\footnote{\url{https://partkeepr.org/}} offers essential inventory management tools, such as batch tracking and stock alerts, tailored to managing electronic components but without the scalability or advanced features needed for more extensive operations.

The following Table \ref{tab:ims_comparison} summarises the key features of these systems, emphasising their areas of strength and limitations:

\begin{table}[!htb]
\centering
\caption{Summarised comparison of open source \acl{ims} by features}
\begin{tabular}{lllllll}
\hline
\textbf{\ac{ims}} & \textbf{MWS} & \textbf{BI} & \textbf{DF} & \textbf{BT} & \textbf{ERPI} & \textbf{UFI} \\
\hline
Odoo & Yes & Yes & Yes & Yes & Yes & Yes \\
\hline
Snipe-IT & No & No & No & No & No & Yes \\
\hline
InvenTree & No & Yes & No & Yes & No & Yes \\
\hline
SkuNexus & Yes & Yes & Yes & Yes & Yes & No \\
\hline
PartKeepr & No & Yes & No & Yes & No & No \\
\hline
\end{tabular}
\caption*{\\MWS - Multi-Warehouse Support, BI - Barcode Integration, DF - Demand Forecasting, BT - Batch Tracking, ERPI - Enterprise Resource Planning Integration, UFI - User-Friendly Interface.}
\label{tab:ims_comparison}
\end{table}

%% ____________________ Lost-And-Found Management Systems ____________________ %%

\section{In-Production \acl{lfms}} \label{sec:in-production-solutions}

The management of \ac{lf} items has evolved over the years, resulting in modern \acp{lfms} that leverage innovative features. This section delves into some of the most prominent \acp{lfms}, grouped by their functionality and strengths. All the systems that are going to be explored combine with integrated shipping and return subsystems, which expresses the importance of this feature in the context of \ac{lf} management.

\subsubsection{Comprehensive and Feature-Rich Solutions} \label{subsubsec:comprehensive-solutions}

NotLost\footnote{\url{https://notlost.com}} stands out as a highly versatile \ac{lfms}, offering a broad array of features that make it suitable for organisations of all sizes. Its strong automated matching and search capabilities, powered by some rudimentary artificial intelligence, simplify item identification and matching processes. The platform also emphasises data security and compliance, but despite its many strengths, NotLost lacks features for disposal and recycling, leaving room for improvement in managing unclaimed items.

Similarly, Chargerback\footnote{\url{https://www.chargerback.com}} offers an extensive feature set comparable to NotLost, with added emphasis on reporting and analytics and disposal and recycling management, making it an ideal choice for organisations that require comprehensive reporting tools to analyse \ac{lf} trends. Chargerback also has the best-found training and support, featuring online training and support, tech support, innovation support and a 24-hour quick response system for partners, which helps organisations onboard their staff effectively.

\subsubsection{Specialised Solutions for Targeted Needs} \label{subsubsec:specialised-solutions}

For organisations seeking user-centric platforms, iLost\footnote{\url{https://ilost.co}} and FoundHero\footnote{\url{https://foundhero.com}} offer intuitive, user-friendly interfaces that facilitate easy reporting and claiming of \ac{lf} items. While iLost shines in its focus on simplicity and efficiency for smooth item recovery, it lacks advanced features like automated matching and analytics. On the other hand, FoundHero emphasises customer feedback collection, which enables organisations to gather valuable insights from users on their \ac{lf} experiences.

Crowdfind\footnote{\url{https://www.crowdfind.com}} also prioritises usability, with a strong emphasis on visual tools such as photo-driven item searches. Its scalability and adaptability across multiple sectors make it a preferred choice for organisations managing large-scale \ac{lf} operations.

\subsubsection{Industry-Specific Solutions} \label{subsubsec:industry-specific-solutions}

ILeftMyStuff\footnote{\url{https://www.ileftmystuff.com}} caters specifically to the hospitality industry, offering specialised tools like automated guest communication via its communication tools. Its focus on these features, along with training and support, secures that hotels and similar establishments can manage \ac{lf} items without massive complaints and the need for intensive learning. However, the platform lacks support for advanced analytics and customer engagement features, limiting its broader applicability.

\subsubsection{Community-Driven and Volunteer-Based Solutions} \label{subsubsec:community-driven-solutions}

In contrast to enterprise-focused solutions, LostMyStuff\footnote{\url{http://www.lostmystuff.net/}} takes a community-driven approach. The platform connects individuals with volunteers to aid in recovering \ac{lf} items. While it lacks advanced technological features, its focus on volunteer and community engagement makes it unique in fostering a sense of shared responsibility and collaboration among users.

\subsubsection{Summary} \label{subsubsec:lfms_summary}

The Table \ref{tab:lfms_features} summarising the features of these \ac{lfms} platforms provides a detailed comparison of their features and capabilities, highlighting the strengths and limitations of each solution.

\begin{table}[!htb]
\centering
\caption{Feature Availability in \acl{lfms}s}
\begin{tabular}{lllllllllllll}
    \toprule
    {} & ILIM & AMS & UFI & CT & ISR & DSC & SMSA & DRM & RA & VCE & CFC & TS \\
    \midrule
    NotLost & Yes & Yes & Yes & Yes & Yes & Yes & Yes & No & Yes & - & - & - \\
    iLost & Yes & No & Yes & Yes & Yes & - & - & No & - & - & No & - \\
    ILeftMyStuff & Yes & - & Yes & Yes & Yes & - & - & No & - & No & - & Yes \\
    ReclaimHub & Yes & No & Yes & Yes & Yes & - & No & Yes & Yes & No & - & No \\
    Crowdfind & Yes & Yes & Yes & Yes & Yes & - & Yes & - & No & No & - & - \\
    Chargerback & Yes & Yes & Yes & Yes & Yes & - & Yes & Yes & Yes & No & No & Yes \\
    MissingX & Yes & No & Yes & Yes & Yes & - & Yes & Yes & Yes & - & - & - \\
    FoundHero & Yes & No & Yes & Yes & Yes & Yes & - & - & Yes & - & Yes & - \\
    LostMyStuff & Yes & - & Yes & - & Yes & - & - & No & - & Yes & - & No \\
    \bottomrule
\end{tabular}
\caption*{\\ILIM - Item Logging and Inventory Management, AMS - Automated Matching and Search, UFI - User-Friendly Interfaces, CT - Communication Tools, ISR - Integrated Shipping and Returns, DSC - Data Security and Compliance, SMSA - Scalability and Multi-Sector Adaptability, DRM - Disposal and Recycling Management, RA - Reporting and Analytics, VCE - Volunteer and Community Engagement, CFC - Customer Feedback Collection, TS - Training and Support.}
\label{tab:lfms_features}
\end{table}

%% ____________________ Designing an Intelligent IMS for LF ____________________ %%

% \section{Designing an Intelligent \acl{ims} for \acl{lf}} \label{sec:designing-intelligent-ims}

%% Item caracterization
%% NLU
%% Object recognition
%% Privacy and security
%% Introduction into IMS
%% User experience and gamification in the context of lost items

%% ____________________ Summary ____________________ %%

% \section{Summary} \label{sec:chapter2-summary}

% Todo: Write this section.
