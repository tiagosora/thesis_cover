\chapter{Discussion and Conclusions}
\label{chapter:discussion_conclusions}

\begin{introduction}
"The things we lose have a way of coming back to us in the end, if not always in the way we expect" \textbf{- J.K. Rowling, Harry Potter and the Order of the Phoenix (2003)}
\end{introduction}

This chapter evaluates research outcomes against objectives, presents contributions to \ac{lf} management, discusses limitations, and outlines future directions.

% ____________________ Research Objectives Achievement ____________________ %

\section{Research Objectives Achievement} \label{section:objectives_achievement}

The research addressed all objectives from Chapter \ref{chapter:introduction}. UAchado implements an intelligent \ac{ims} using \ac{llava} for automated identification and categorisation. The matching algorithm integrates multiple factors in a weighted confidence framework, achieving high accuracy with 20-second processing times. Automated categorisation reaches 92\% accuracy with fallback mechanisms for service unavailability.

User testing validated the dual-interface approach (React dashboard for administrators, Flutter mobile for users), since task completion rates were high across all scenarios, with administrative staff achieving perfect completion rates. Security guards unanimously preferred the digital solution over paper methods.

Community features were partially implemented through an architecture enabling independent community operation with shared infrastructure. The database supports future messaging systems, though direct communication features were deferred to prioritise core functionality.

Testing reached 46\% code coverage across all functions. Performance tests showed 200 concurrent connections with P95 latency of 15.83ms. Prometheus and Grafana provide monitoring through 24 metrics across 5 dashboards. The queue-based architecture allows A/B testing without service disruption.

Overall, most of the objectives previously outlined in the Chapters \ref{chapter:introduction}, \ref{chapter:state_of_art} and \ref{chapter:methodology} were met within the framed timeline. The system reached 92.5\% production confidence with exceptional performance improvements. The modular monolithic architecture worked well for academic contexts while preserving scalability.

% ____________________ Research Contributions ____________________ %

\section{Research Contributions} \label{section:research_contributions}

This work contributes to intelligent lost property management through theoretical advances, practical implementation, and methodological approaches.

The \ac{slr} in Chapter \ref{chapter:state_of_art} analysed 476 studies following \ac{prisma} guidelines, integrating 18 papers to establish a structured analysis of \ac{ai} in lost property management. The review identified \ac{yolo} and \ac{resnet}-50 as dominant architectures, while revealing challenges in computational intensity (26\%), dataset quality (26\%), and scalability (21\%). It revealed gaps in multimodal matching and community solutions, supporting this research approach.

The novel matching algorithm combines multiple factors in a weighted confidence framework with a binary threshold at 0.90. It goes beyond single-modality approaches while maintaining transparency through interpretable scores and fallback mechanisms.

% ____________________ Limitations and Future Opportunities ____________________ %

\section{Limitations and Future Opportunities} \label{section:limitations_opportunities}

Although the research achieved its objectives, several limitations emerged during development, providing insights for future research.

External \ac{ai} service dependency on Ollama/\ac{llava} affects autonomy, requiring network connectivity (in the tested environment to skynet.av.it.pt/ollama). Processing averages 2.3 seconds per image, potentially bottlenecking under high load despite queue-based mitigation.

PocketBase's SQLite backend limits scalability due to single-writer constraints. The 17-collection schema may require restructuring for additional features. The 10MB file limit suits mobile uploads but may not accommodate all documentation types.

Authentication lacks integration with Lightweight Directory Access Protocol (LDAP), Active Directory, or Security Assertion Markup Language (SAML) systems, the three-tier role system may be insufficient for larger hierarchies and the token-based synchronisation between platforms may not scale for large populations.

The implementation embeds University of Aveiro-specific assumptions. Fixed proximity thresholds (1km, 5km or 20km, at maximum) may not suit other geographies. Portuguese is the only secondary language supported, limiting accessibility for non-English and non-Portuguese speakers.

In terms of validation, testing offered limited demographic coverage and the performance testing on a single environment (MacBook Air M2 with Apple M2 chip, 8-core CPU, 8-core GPU) may not represent production conditions. Though the tested code was specifically designed for this environment, incomplete code coverage leaves some error paths and edge cases untested.

Fixed algorithm weights may not work best across all contexts and the confidence thresholds lack empirical tuning. The visual matching also depends on image quality often inconsistent in real-world scenarios and this factor was not deeply studied in this research.

The \ac{nlp} processing, primarily optimised for English, presents limitations in multilingual contexts. Similarly, the predefined taxonomies may not accommodate the diverse patterns of lost items across different institutional environments. While these linguistic and classification constraints were identified during development, addressing them fell outside the dissertation's primary scope.

\subsection{Future Research Opportunities}

Future technical research should focus on local \ac{ai} deployment through edge computing and federated learning to reduce external dependencies. Local deployment using optimized inference frameworks could eliminate network latency while providing notable performance improvements over current external service dependencies.

Domain-specific architectures with real-time adaptation based on human feedback could improve matching accuracy and efficiency. Some implementation approaches include Redis-based caching layers for frequently accessed data, connection pooling optimizations targeting 5,000+ concurrent connections, and in-memory session stores to reduce database load. These optimizations could potentially increase throughput from the current 767 \ac{rps} to 3,000-5,000 \ac{rps} for standard operations.

From an infrastructure perspective, distributed architectures and microservices would address current scalability limitations. Specific implementations could also include horizontal scaling using Kubernetes orchestration. Load balancing across multiple application instances could support 10000+ concurrent users while maintaining sub-100ms response times. Besides that, container orchestration with auto-scaling based on CPU and memory metrics would provide elastic resource allocation during peak usage periods.

A testing approach using AWS cloud infrastructure would provide realistic and more reliable production performance metrics beyond current local development constraints. The proposed testing framework should deploy the system using AWS ECS or EKS for container orchestration, RDS PostgreSQL for scalable database operations, and ElastiCache Redis for caching layers. Load testing could be implemented using distributed tools such as Artillery.io or k6 deployed across multiple AWS regions, or cloud-native solutions like AWS Load Testing solution built on Amazon ECS. These tools could simulate from 1000 to 10000 concurrent users with realistic traffic patterns including ramp-up phases, sustained load periods, and stress testing scenarios. CloudWatch monitoring integrated with custom Grafana dashboards would capture detailed performance metrics including response times, throughput, error rates, CPU utilization, memory consumption, network I/O, and database query performance. This would enable testing scenarios impossible in local environments.

% ____________________ Conclusions and Future Work ____________________ %

\section{Conclusions and Future Work} \label{section:conclusions_future_work}

This dissertation transformed traditional lost property management through intelligent technologies, demonstrating that \ac{ai}-powered systems can outperform traditional approaches while maintaining operational reliability. UAchado met the primary objective by integrating multimodal \ac{ai} with architectural principles, achieving performance validation results that showed response times 33x better than targets with 94.2\% matching accuracy. The system processes 156 images per hour at 85\% utilisation, while user testing confirmed 88\% overall task completion rates, with 92\% success for item reporting.

The research made several theoretical and practical contributions to the field. The \ac{prisma}-compliant \ac{slr} established an academic foundation for intelligent \ac{lf} management, identifying gaps that this work addressed during implementation. The multimodal algorithm advances beyond single-modality solutions with interpretable confidence scoring. The architectural pattern balances organisational isolation with resource efficiency through role-based control, while the production framework bridges research prototypes and operational deployment through fallback mechanisms and monitoring infrastructure.

The implications for practice and research are important. Testing results indicate organisations can expect measurable improvements in processing speed, matching accuracy, and user satisfaction when adopting intelligent \ac{lfms}. Future research directions include exploring local \ac{ai} deployment through edge computing and federated learning, investigating distributed architectures for enhanced scalability, and examining blockchain technologies for cross-institutional networks. Additional opportunities exist in accessibility features, gamification strategies, privacy-preserving techniques, and long-term studies of adoption patterns and social impacts.

Throughout this research, the human dimension of lost property management has remained central to every technical decision and architectural choice. As mentioned in the opening quote of Chapter \ref{chapter:introduction}, Dinah Maria Craik once observed that "we never discover the value of things till we have lost them". UAchado transforms this fundamental human experience by making loss recoverable. Where once a misplaced item faced uncertain odds of recovery through paper-based systems, intelligent technology now provides a 94.2\% chance of successful matching, turning hope into statistical probability.

The journey from initial problem recognition to production deployment demonstrates that academic research, when grounded in real-world constraints and human needs, can bridge the gap between theoretical possibility and practical impact. UAchado's success at the University of Aveiro proves that effective \ac{ai} integration need not require vast resources or lengthy development cycles, but thoughtful engineering, rigorous testing, and unwavering focus on user needs. As the opening quote from J.K. Rowling reminds us, "the things we lose have a way of coming back to us in the end, if not always in the way we expect", UAchado represents that unexpected technological solution to an age-old human problem, transforming the experience of loss into an opportunity for community connection and technological innovation.

