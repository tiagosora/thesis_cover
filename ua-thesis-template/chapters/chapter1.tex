\chapter{Introduction}
\label{chapter:introduction}

\begin{introduction}
"We never discover the value of things till we have lost them"

- Dinah Maria Craik, A Life For A Life (1859)
\end{introduction}

\section{Context} \label{section:context}

The loss of personal items is a common situation that affects individuals in various contexts, from public spaces such as airports, universities and shopping centres \cites{Oke2017, Yao2019} to private institutions such as schools, companies and factories. On average, people misplace or completely lose up to nine items each week, with common culprits including mobile phones, keys, and sunglasses \cite{Prawira2024}. This distraction costs individuals from 15 to 50 minutes each day spent searching for these misplaced items \cites{Prawira2024, Knierim2012}. Over an average of 60 years of stated adult life, that implies approximately at least a total of 3680 hours (153,3 days) and 200000 items misplaced or wholly lost \cite{Ahmad2015}. Adding to this issue, studies have revealed that the time spent searching for lost items can lead to financial losses that feel akin to literally throwing money away. The cumulative effect of these lost hours not only frustrates individuals but also impacts their financial well-being, a stark reminder of how daily distractions can drain both time and resources. These effects, when aggregated over long periods, also have alarming outcomes. The same studies also reveal a staggering annual waste of approximately \$177 billion dollars \cite{Newswire2010} due to time spent searching for lost or misplaced items \cite{Ahmad2015}, which represents a figure that highlights a significant drain on productivity.

Traditionally, manually managing lost property has long been plagued by inefficiencies that disadvantage its corresponding stakeholders, i.e. the administrators and the individuals, both seeking to deliver and recover the lost belongings \cite{Sinha2024}. In most cases, the process relies heavily on manual efforts, requiring staff to record, store and track items using rudimentary tools such as paper records or basic spreadsheets. Others resort to simple and outdated \acp{ims} that are not designed to handle the complexities of lost property management \cite{Guinard2008}. This labour-intensive approach is time-consuming and prone to human error, leading to lost items, inaccurate records and miscommunication between departments or stakeholders \cite{Sinha2024, Guinard2008}. Additionally, most of the designated staff responsible for handling these tasks is rarely compensated or formally recognised for this additional responsibility. These employees are typically expected to manage lost items alongside their regular workload without extra pay, training, or resources \cite{Guinard2008}. They are tasked with organising found objects, responding to enquiries and ensuring that the rightful owners are identified, often with little or no support from automated or systematic processes that divert attention from their core responsibilities and also foster frustration as they navigate an unsustainable workflow.

On the other hand, individuals searching for lost objects face significant challenges. The lack of a standardised or intuitive system means they have to rely on guesswork or luck to recover their belongings. The stress is even more significant when they do not know where to start their search or whether their lost object has even been found. The so-called "lost-and-found" effect is described by \citeauthoryear{Garling2023}, explaining the stressful mental process of over-valuing and prioritising the recovery of a missing item.

The absence of transparent communication channels \cite{Guinard2008} or efficient recovery mechanisms further enhances this effect. Complementing the ineffectiveness, the risks associated with poorly managed lost items are substantial. Without secure processes, lost objects are vulnerable to theft or unauthorised access \cite{Tan2023}. Identity forgery becomes a tangible threat when personal or sensitive belongings, such as identification documents or electronic devices, fall into the wrong hands \cite{Xue2022}. In addition, the public disclosure of personal information associated with lost property has become a concern that has not yet been the subject of a standardised solution \cite{Xue2022}. Furthermore, mismanagement can lead to legal complications, especially if disputes arise over lost objects that are improperly registered or returned to the wrong person. A prevalent issue with lost property platforms is the failure to establish a robust sense of trust among users \cite{Xue2022}. Specifically, there is often a lack of reliable assurance between the owner of the lost item and the finder. This vulnerability can lead to dishonest behaviour, including fraudulent claims or requests.

From an organisational point of view, even though some institutions provide designated \ac{lf} collection points \cite{Tan2023}, they also run the risk of damaging their reputation, potentially being held accountable and damaging relations with the communities they serve.

In a world where technology effortlessly simplifies our daily tasks, it is striking that we lack effective technological solutions for locating our physical belongings. This discrepancy highlights a critical gap in our everyday lives, underscoring the urgent need for innovative tools that can help us find our lost possessions efficiently.

\section{Motivation} \label{section:motivation}

Since managing lost objects still represents a common challenge, creating more efficient systems to solve this problem can ease the burden among users and the community. With more efficient lost property information systems, manual processes that are currently riddled with inefficiencies, such as time-consuming workflows, human error and a lack of transparent communication channels, can be completely redesigned. Institutions can not only optimise recovery processes but also strengthen their commitment to user satisfaction, thus improving community relations and institutional reputation.

Emerging technologies such as \ac{ai}, \ac{nlp} and the increasing training and use of \acp{llm} offer unprecedented opportunities to address these challenges. \ac{ai} can facilitate the identification and categorisation of lost objects, while \ac{nlp} enables the use of users' interactions, such as searches and conversations, to extract relevant data \cite{Prawira2024}. In recent years, the advent of deep learning has revolutionised the performance of various visual tasks, leading to significant advancements in areas such as image classification \cite{Liu2022}. These improvements are not limited to mere accuracy; they encompass enhanced efficiency in processing large datasets, the ability to recognise intricate patterns, and the capability to generalise across diverse scenarios. This thesis is motivated by the potential of these technologies to innovate in an area that remains underexplored, aiming to add to the lost object management into an efficient, secure and user-friendly experience. The integration of these technologies can redefine the standards of lost object recovery and management, resolving inefficiencies and building trust among stakeholders. Beyond merely improving operational processes, a reimagined system for managing lost property must also emphasise community integration. By leveraging technologies to enable direct communication between finders and owners, supported by platforms for community-based reporting, the solution can minimise reliance on intermediaries and streamline the recovery process. The community-oriented approach proposed by \citeauthoryear{Guinard2008} not only fosters trust among stakeholders but also encourages a culture of shared responsibility and collaboration, empowering users to take an active role in solving lost-and-found challenges.

A well-designed lost property management system has the potential to extend its impact far beyond the immediate issue of misplaced belongings. By addressing the inefficiencies inherent in traditional processes and adopting a community-driven, technologically sophisticated approach, the proposed solution could serve as a model for innovation in related domains, offering a framework for reducing theft, enhancing data privacy, and fostering transparency, which can, in turn, bolster trust. Moreover, such a system represents a tangible demonstration of how intelligent technologies can address real-world problems, ultimately contributing to a more connected, efficient, and equitable society. In this way, solving the problem of lost property management not only resolves a long-standing challenge but also sets a precedent for the transformative power of technological solutions.

\section{Objectives} \label{section:objectives}

% TODO: Mudar para a produção de um ILFMS

This research seeks to establish a secure and straightforward framework for addressing the inefficiencies in the management of lost property by defining specific and measurable objectives. A primary goal is to design and develop an \ac{ims} by exploring the application and capabilities of \ac{ai} and \ac{nlp} for automating the identification, categorisation, and recommendation of lost items, aiming to reduce manual workload and effectively complete the \ac{lf} cycle of items. Another objective is to prioritise usability by ensuring that the proposed solution accommodates individuals with varying levels of technological proficiency, which involves examining design strategies that promote accessibility and simplify interactions, fostering a more inclusive approach. Encouraging community engagement is also a key focus, with an emphasis on fostering direct communication between users to strengthen collaboration and reduce dependency on intermediaries. This engagement indicates that the level of user interactions will highly modify the user experience.

The research aims to employ iterative testing and validation in both controlled and real-world settings, including evaluating the system's performance, usability, and scalability under diverse conditions. Consequentially, it implies the need to implement monitoring and observability methods to generate actionable insights, enabling continuous improvement of the solution.

The proposed research can be conducted in four key stages, which align with a logical progression from understanding the problem to delivering a robust, focusing on different aspects of design and development.

\begin{enumerate}
    \item Study the State-of-the-Art: Review existing i) on-production solutions, ii) available and recommended technologies, iii) best practices and optimal approaches and iv) challenges;
    \item Design the structure of the system by defining its architecture, features and workflows;
    \item Create the main system components, including all the planned features;
    \item Test the system in a real and controlled environment;
    \item Finalise and document the results for academic and practical use.
\end{enumerate}

\section{Dissertation Outline} \label{section:dissertation_outline}

This dissertation is structured into multiple chapters, each building upon the previous to provide a comprehensive understanding of the project and its outcomes. The document begins with an introductory chapter, Chapter \ref{chapter:introduction}, that offers background information, including the core problem being addressed, the motivation for this research, and the objectives associated with the proposed solution.

Following this, Chapter \ref{chapter:state_of_art} examines the state of the art by reviewing the current literature and relevant technologies and on-production solutions. This chapter explores topics such as ... % TODO: Ver o que vou explorar

Chapter \ref{chapter:methodolody} ... % TODO: Ver o que vou explorar

Chapter 4 ... % TODO: Ver o que vou explorar

In Chapter 5, ... % TODO: Ver o que vou explorar

Finally, Chapter 6 will conclude the dissertation by summarizing the key outcomes and contributions of this work to the field. % TODO: Ver o que vou explorar
