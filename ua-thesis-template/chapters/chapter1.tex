\chapter{Introduction}
\label{chapter:introduction}

\begin{introduction}
"We never discover the value of things till we have lost them"

\textbf{- Dinah Maria Craik, A Life For A Life (1859)}
\end{introduction}

% \section{Context} \label{section:context}

The loss of personal items is a common situation that affects individuals in various contexts, from public spaces such as airports, universities and shopping centres \cites{Oke2017, Yao2019} to private institutions such as schools, companies and factories.
% On average, people misplace or completely lose up to nine items each week, with common culprits including mobile phones, keys, and sunglasses \cite{Prawira2024}. This distraction costs individuals between 15 and 50 minutes each day spent searching for these misplaced items \cites{Prawira2024, Knierim2012}. Over an average of 60 years of stated adult life, that implies approximately at least a total of 3680 hours (153,3 days) and 200 000 items misplaced or wholly lost \cite{Ahmad2015}. Adding to this issue, studies have revealed that the time spent searching for lost items can lead to financial losses that feel akin to throwing money away. The cumulative effect of these lost hours not only frustrates individuals but also impacts their financial well-being, a stark reminder of how daily distractions can drain both time and resources. These effects, when aggregated over long periods, also have alarming outcomes. The same studies also reveal a staggering annual waste of approximately \$177 billion dollars made by United State citizens \cite{Newswire2010} due to time spent searching for lost or misplaced items \cite{Ahmad2015}, which represents a figure that spotlights a significant drain on productivity.

Traditionally, manually managing lost property has long been plagued by inefficiencies that disadvantage its corresponding stakeholders, i.e. the administrators and the individuals, both seeking to deliver and recover the lost belongings \cite{Sinha2024}. In most cases, the process relies heavily on manual efforts, requiring staff to record, store and track items using rudimentary tools such as paper records or basic spreadsheets. Others resort to simple and outdated \acp{ims} that are not designed to handle the complexities of lost property management \cite{Guinard2008}. This labour-intensive approach is time-consuming and prone to human error, leading to lost items, inaccurate records and miscommunication between departments or stakeholders \cite{Sinha2024, Guinard2008}. Additionally, most of the designated staff responsible for handling these tasks is rarely compensated or formally recognised for this additional responsibility. These employees are typically expected to manage lost items alongside their regular workload without extra pay, training, or resources \cite{Guinard2008}. They are tasked with organising found objects, responding to enquiries and ensuring that the rightful owners are identified, often with little or no support from automated or systematic processes that divert attention from their core responsibilities and also foster frustration as they navigate an unsustainable workflow.

On the other hand, individuals searching for lost objects face many significant challenges. The lack of a standardised or intuitive system means they have to rely on guesswork or luck to recover their belongings. The stress is even more significant when they do not know where to start their search or if their lost object has already been found. It is the so-called "lost-and-found" effect, described by \citeauthoryear{Garling2023}, explaining the stressful mental process of over-valuing and prioritising the recovery of a missing item. The absence of transparent communication channels \cite{Guinard2008} or efficient recovery mechanisms further enhances this effect.

Complementing the ineffectiveness, the risks associated with poorly managed lost items are substantial. Without secure processes, lost objects are vulnerable to theft or unauthorised access \cite{Tan2023}. Identity theft and forgery become tangible threats when personal or sensitive belongings, such as identification documents or personal electronic devices, fall into the wrong hands \cite{Xue2022}. In addition, the public disclosure of personal information associated with lost property has become a concern that has not yet been the subject of a standardised solution \cite{Xue2022}. Furthermore, mismanagement can lead to legal complications, especially if disputes arise over lost objects that are improperly registered or not returned to the rightful owner. A prevalent issue with lost property platforms is the failure to establish a robust sense of trust among users \cite{Xue2022}. Specifically, there is often a lack of reliable assurance between the owner of the lost item and the finder. This vulnerability can lead to dishonest behaviour, including fraudulent claims or requests.

From an organisational point of view, even though some institutions provide designated \ac{lf} collection points \cite{Tan2023}, they also run the risk of damaging their reputation, potentially being held accountable and damaging relations with the communities they serve.

In a world where technology effortlessly simplifies our daily tasks, it is striking that we lack effective technological solutions for locating our physical belongings. This discrepancy highlights a critical gap in our everyday lives, underscoring the urgent need for innovative tools that can help us find our lost possessions efficiently. Advancements in \ac{ai}-based technologies, such as \ac{nlp}, and the increasing training and use of \acp{llm} offer unprecedented opportunities to address these challenges. \ac{ai} can facilitate the identification and categorisation of lost objects, while \ac{nlp} enables the use of users' interactions, such as searches and conversations, to extract relevant data \cite{Prawira2024}, which can then be used in the user's best interest. In recent years, the advent of deep learning has revolutionised the performance of various visual tasks, leading to significant advancements in areas such as image classification \cite{Liu2022}. These improvements are not limited to mere accuracy; they encompass enhanced efficiency in processing large datasets, the ability to recognise intricate patterns, and the capability to generalise across diverse scenarios. 

\section{Motivation} \label{section:motivation}

On average, people misplace or completely lose up to nine items each week, with common culprits including mobile phones, keys, and sunglasses \cite{Prawira2024}. This distraction costs individuals between 15 and 50 minutes each day spent searching for these misplaced items \cites{Prawira2024, Knierim2012}. Over an average of 60 years of stated adult life, that implies approximately at least a total of 3680 hours (153,3 days) and 200 000 items misplaced or wholly lost \cite{Ahmad2015}. Adding to this issue, studies have revealed that the time spent searching for lost items can lead to financial losses that feel akin to throwing money away. The cumulative effect of these lost hours not only frustrates individuals but also impacts their financial well-being, a stark reminder of how daily distractions can drain both time and resources. These effects, when aggregated over long periods, also have alarming outcomes. The same studies also reveal a staggering annual waste of approximately \$177 billion dollars made by United States citizens \cite{Newswire2010} due to time spent searching for lost or misplaced items \cite{Ahmad2015}, which represents a figure that spotlights a significant drain on productivity.

Since managing lost objects still represents a common challenge, creating a more efficient system to solve this problem can ease the burden among users and the community. With more efficient \ac{lfms}, manual processes that are currently riddled with inefficiencies can be completely redesigned. Institutions can not only optimise recovery processes but also strengthen their commitment to user satisfaction, thus improving community relations and institutional reputation.

% Advancements in \ac{ai}-based technologies, such as \ac{nlp}, and the increasing training and use of \acp{llm} offer unprecedented opportunities to address these challenges. \ac{ai} can facilitate the identification and categorisation of lost objects, while \ac{nlp} enables the use of users' interactions, such as searches and conversations, to extract relevant data \cite{Prawira2024}, which can then be used in the user's best interest. In recent years, the advent of deep learning has revolutionised the performance of various visual tasks, leading to significant advancements in areas such as image classification \cite{Liu2022}. These improvements are not limited to mere accuracy; they encompass enhanced efficiency in processing large datasets, the ability to recognise intricate patterns, and the capability to generalise across diverse scenarios. 

This dissertation is motivated by the potential of these technologies to innovate in an area that remains barely explored, aiming to transform the \ac{lf} management into an efficient, secure and user-friendly experience. The integration of these technologies can redefine the standards of lost objects' recovery and management, resolving inefficiencies and building trust among stakeholders. Beyond merely improving operational processes, a reimagined system for managing lost property must also emphasise community integration. By enabling direct communication between finders and owners, supported by platforms for community-based reporting, the solution can minimise reliance on intermediaries and streamline the recovery process. The community-oriented approach proposed by \citeauthoryear{Guinard2008} not only boosts trust among stakeholders but also encourages a culture of shared responsibility and collaboration, empowering users to take an active role in solving \ac{lf} challenges.

% A well-designed \ac{lfms} has the potential to extend its impact far beyond the immediate issue of misplaced belongings. By addressing the inefficiencies inherent in traditional processes and adopting a community-driven and technologically sophisticated approach, the proposed solution could serve as a model for innovation in related domains, offering a framework for reducing theft, enhancing data privacy, and fostering transparency, which can, in turn, bolster trust. Moreover, such a system represents a tangible demonstration of how intelligent technologies can address real-world problems, ultimately contributing to a more connected, efficient, and equitable society. In this way, solving the problem of lost property management not only resolves a long-standing challenge but also sets a precedent for the transformative power of technological solutions.

\section{Objectives} \label{section:objectives}

This research seeks to establish a framework for addressing the inefficiencies in the management of \ac{lf} property by defining specific and measurable objectives. A primary goal is to design and develop an intelligent \ac{ims} specialised for \ac{lf} by exploring the application and capabilities of \ac{ai}-based technologies, such as \ac{cv} for automating the identification and categorisation, and \ac{nlp} and matching and recommendation of lost items, aiming to reduce manual workload and effectively complete the \ac{lf} cycle of items. Another objective is to prioritise usability by ensuring that the proposed solution accommodates individuals with varying levels of technological proficiency, which involves examining design strategies that promote accessibility and simplify interactions, fostering a more inclusive approach. The last objective is to include community-oriented features, which may encourage community engagement, emphasising promoting direct communication between users. This engagement indicates that the level of user interactions will highly modify the user experience.

The research aims to employ iterative testing and validation in both controlled and real-world settings, including evaluating the system under diverse conditions. Consequentially, it implies the need to implement record-keeping and observability methods to generate actionable insights, enabling continuous improvement of the solution.

% The proposed research can be conducted in four key stages, which align with a logical progression from understanding the problem to delivering a robust, focusing on different aspects of design and development.

% \begin{outline}[enumerate]
%     \1 Study the State-of-the-Art and identify design approaches:
%         \2 Review existing ;
%         \2 \acl{lfms};
%         \2 In-production solutions;
%         \2 Available and recommended technologies;
%         \2 Best practices and optimal approaches;
%     \1 Design the structure of the system by defining its architecture, features and workflows;
%     \1 Create the main system components, including all the planned features;
%     \1 Test the system in a real-world and controlled environment;
%     \1 Finalise and document the results for academic and practical use.
% \end{outline}

\section{Dissertation Outline} \label{section:dissertation_outline}

This dissertation is structured into six chapters, each building upon the previous to provide a complete understanding of the research and its outcomes. The document begins with this introductory chapter, Chapter \ref{chapter:introduction}, that offers background information, including the core problem being addressed, the motivation for this research, and the objectives associated with the proposed solution.

Chapter \ref{chapter:state_of_art} examines the state of the art and explores topics such as traditional and modern \acp{lfms}; \ac{cv}, directly connected with the advances on \ac{ai}; \ac{nlp} and the use of \acp{llm} to perform \ac{nlu} tasks; and related technologies.

Chapter \ref{chapter:methodology} describes the methodology and design decisions for the platform; architectural design approach; system requirements; architectural evolution and implementation decisions; and the development framework guiding implementation.

Chapter \ref{chapter:implementation} covers system development; core functionality implementation; database design and collections structure; \ac{ai} integration with LLaVA; deployment architecture; API documentation; and technical implementation decisions.

Chapter \ref{chapter:testing_validation} describes validation and testing procedures; evaluation methodology; test scenarios; performance testing and validation; and effectiveness assessment in controlled and real-world environments.

Chapter \ref{chapter:discussion} discusses results; limitations encountered; contributions made; conclusions drawn from this research; and suggestions for future work.

Additionally, Appendix \ref{app:architectural-design} contains detailed architectural design documentation that supports the technical implementation described in the main chapters.
